% !TEX encoding = UTF-8 Unicode

%----------------------------------------------------------------------------------------
%	PACKAGES AND OTHER DOCUMENT CONFIGURATIONS
%----------------------------------------------------------------------------------------

\documentclass[twoside]{article}

\usepackage[sc]{mathpazo} % Use the Palatino font
\usepackage[utf8]{inputenc}
\usepackage[english]{babel}
\usepackage[T1]{fontenc} % Use 8-bit encoding that has 256 glyphs
%\linespread{1.05} % Line spacing - Palatino needs more space between lines
\usepackage{microtype} % Slightly tweak font spacing for aesthetics

%\usepackage[hmarginratio=1:1,top=32mm,columnsep=20pt]{geometry} % Document margins
\usepackage[top=20mm,left=8mm,right=8mm,bottom=20mm]{geometry}

\usepackage{multicol} % Used for the two-column layout of the document
\usepackage[hang, small,labelfont=bf,up,textfont=it,up]{caption} % Custom capti
\usepackage{booktabs} % Horizontal rules in tables
\usepackage{float} % Required for tables and figures in the multi-column environment - they need to be placed in specific locations with the [H] (e.g. \begin{table}[H])
\usepackage{hyperref} % For hyperlinks in the PDF

\usepackage{lettrine} % The lettrine is the first enlarged letter at the beginning of the text
\usepackage{paralist} % Used for the compactitem environment which makes bullet points with less space between them

\usepackage{abstract} % Allows abstract customization
\renewcommand{\abstractnamefont}{\normalfont\bfseries} % Set the "Abstract" text to bold
\renewcommand{\abstracttextfont}{\normalfont\small\itshape} % Set the abstract itself to small italic text

%2º espacamento antes da section
\usepackage{titlesec} % Allows customization of titles
%\titlespacing*{\section}{0pt}{0pt}{0pt}
\renewcommand\thesection{\Roman{section}} % Roman numerals for the sections
\renewcommand\thesubsection{\Roman{subsection}} % Roman numerals for subsections
\titleformat{\section}[block]{\large\scshape\centering}{\thesection.}{1em}{} % Change the look of the section titles
\titleformat{\subsection}[block]{\large}{\thesubsection.}{1em}{} % Change the look of the section titles

\usepackage{fancyhdr} % Headers and footers
\pagestyle{fancy} % All pages have headers and footers
\fancyhead{} % Blank out the default header
\fancyfoot{} % Blank out the default footer
\fancyhead[C]{University of Minho $\bullet$ CPD Integrated Project 2015-2016} % Custom header text
\fancyfoot[RO,LE]{\thepage} % Custom footer text

%%% ADDED BY ME %%%%%%
%\floatstyle{boxed} 
\restylefloat{figure}
\usepackage{graphicx}
\usepackage{caption}
\usepackage{subcaption}
\usepackage{amsfonts}
\usepackage{listings,mdframed}
\usepackage{fancyvrb}
\usepackage{cleveref}
\usepackage{mathtools}
\usepackage{amsmath}
\usepackage{color}
\usepackage{relsize}
\usepackage{array}
\newcolumntype{L}[1]{>{\raggedright\let\newline\\\arraybackslash\hspace{0pt}}m{#1}}
\newcolumntype{C}[1]{>{\centering\let\newline\\\arraybackslash\hspace{0pt}}m{#1}}
\newcolumntype{R}[1]{>{\raggedleft\let\newline\\\arraybackslash\hspace{0pt}}m{#1}}
\usepackage{tabularx,caption}
\usepackage{multirow}

%%% For using norm || %%%%%
\newcommand{\norm}[1]{\left\lVert#1\right\rVert}

\newcommand*{\Package}[1]{\texttt{#1}}

\lstset{
    numbers=left,
	language=C,
	keywordstyle=\bfseries\ttfamily\color[rgb]{0,0,1},
	identifierstyle=\ttfamily,
	commentstyle=\color[rgb]{0.133,0.545,0.133},
	stringstyle=\ttfamily\color[rgb]{0.627,0.126,0.941},
	showstringspaces=false,
	basicstyle=\scriptsize,
numberstyle=\tiny,
numbers=right,
	stepnumber=1,
	numbersep=10pt,
	tabsize=1,
	breaklines=true,
	prebreak = \raisebox{0ex}[0ex][0ex]{\ensuremath{\hookleftarrow}},
	breakatwhitespace=false,
	aboveskip={1.5\baselineskip},
  columns=fixed,
  upquote=true,
  extendedchars=true,
 frame=single,
 inputencoding=utf8,
    literate={á}{{\'a}}1 {ã}{{\~a}}1 {â}{{\~a}}1 {é}{{\'e}}1 {ê}{{\'e}}1 {ç}{{\'c}}1 {ú}{{\'u}}1 {ó}{{\'o}}1 {í}{{\'i}}1,
 %backgroundcolor=\color{lbcolor},
}

\usepackage{blkarray}
\usepackage{stmaryrd}

%----------------------------------------------------------------------------------------
%	TITLE SECTION
%----------------------------------------------------------------------------------------

\title{\vspace{-15mm}\fontsize{24pt}{10pt}\selectfont\textbf{Optimisation of a Linear Algebra Approach to OLAP}} % Article title

\author{
\large
\textsc{Filipe Oliveira} - \textsc{A57816}\\
\normalsize \href{mailto:a57816@alunos.uminho.pt}{a57816@alunos.uminho.pt}
\vspace{-5mm}
\and
\textsc{Sérgio Caldas} - \textsc{A57779}\\
\normalsize \href{mailto:a57779@alunos.uminho.pt}{a57779@alunos.uminho.pt}
}



%----------------------------------------------------------------------------------------

\begin{document}

\maketitle % Insert title

\thispagestyle{fancy} % All pages have headers and footers

%----------------------------------------------------------------------------------------
%	ABSTRACT
%----------------------------------------------------------------------------------------

\begin{abstract}
\indent 
\par Online Analytical Processing (OLAP) systems, perform multidimensional analysis of business data and provides the capability for complex calculations, trend analysis, and sophisticated data modelling. 
Prior efforts have been made to prove that linear algebra (LA) is better suited than standard relational algebra(RA) for formalising and implementing queries in on-line multidimensional data analysis \cite{macedo2015linear} \cite{da2015benchmarking}, however, further work needed to be develop in order to specify and optimize parallel execution of the LA workflow.\par 
The proposed solution focus on a parallel typed linear algebra approach, enumerating how the Linear Algebra operations are efficiently implemented, giving preliminary experimental results obtained with one cluster of Search6 using queries of the TPC-H Benchmark, and comparing the achieved results with a parallel relational algebra engine -- PostgreSQL version 9.6. The first query of the TPC-H Benchmark was fully translated into LA operations to evaluate the performance of the Linear Algebra solution. As these experiments manage large amounts of data, two studies for sparse data representation were made in order to efficiently represent and access information in the LA approach.\par 
 By analysing the query results, we concluded that LA is extremely efficient in executing large data workflows, with benefits as datasets increase versus the RA approach.\par 
\end{abstract}
\vspace{0.5cm}

%----------------------------------------------------------------------------------------
%	ARTICLE CONTENTS
%----------------------------------------------------------------------------------------
\begin{multicols}{2} % Two-column layout throughout the main article text


% !TEX encoding = UTF-8 Unicode

\section{Introduction}
\indent

The design and development of systems that generate, collect, store, process, analyse, and query large sets of data is filled with significant challenges both hardware and software. Combined, these challenges represent a difficult landscape for software engineers.\par 
The relation database is the current solution for big data storage.
Data dependencies in databases can be seen as a binary two-way associative relation. Regarding that lema, 
prior efforts have been made \cite{macedo2015linear} \cite{da2015benchmarking} in the research project "Linear Algebra approach to OLAP", in order to fully represent relational algebra in terms of linear algebra operators.
\par 
OLAP is resource-demanding and calls for parallelisation. Regarding the challenge of High Performance Computing, we implemented from the start a typed linear algebra solution given special importance to the data modelling which, if done poorly, limits the attainable efficiency in data-intesive systems like OLAP that tends to access massive amounts of data and is thus time consuming. \par 
With respect to performance evaluation and results validation in a real work scenario, the used datasets were produced with TPC-H Benchmark, in which is workload consists of multiple query runs.
In order to obtain realistic and meaningful results large datasets were considered, ranging from 1 to 64GB.\par 
In order to infer conclusions and compare relational and linear algebra the object-relational database management system PostgreSQL version 9.6, with roots in open source community, was chosen in order to represent the relational algebra approach. \par 
Given this proximity between database relations and linear algebra, the question arises: does the linear algebra approach presents performance improvements when compared with the relational one?\par 
The report is organised as follows. Section 2 introduces XXXX. Section 3 presents XXXX. Section 4 gives experimental results. Section 5 presents related work. Section 6 concludes.
\section{Staregy}
\indent
\subsection{Towards a linear algebra semantics for SQL}


Aggregation Performance.
Aggregations occur in all TPC-H queries, hence performance of group-by and aggregation is quite important.









Database, application, and storage servers ship with a large number of configu- ration parameters like buffer cache sizes, number of I/O daemons, and parameters input to the database query opti- mizer?s cost model. Finding good settings for these param- eters is a challenging task because of the complex ways in which parameter settings can affect performance

\section{Sequential Experimentation}
\label{sequential}
\indent

Regarding the prior described lemas, we shall now assess wether if the linear algebra approach presents performance improvements when compared with the relational one. 
Before we discuss the measured performance results in the following section, we will briefly summarise characteristics of the multicore platform in our test suite, and present an overview of the performed tunings.\\

Throughout all experiments, the same platform was used. The system, referenced as compute node 652-1, has two Intel\textsuperscript{\textregistered} Xeon\textsuperscript{\textregistered} E5-2670v2 (Ivy Bridge architecture) sharing 64 GB of DDR3 RAM, 1333 MHz, accessed through 4 memory channels. Table \ref{table:characterization} fully characterises the hardware features of the test platform:

\begin{table}[H]
\centering
  \begin{tabular}{ | L{3.5cm} | R{5cm} | }
  
    \hline
    System & compute-652-1 \\ \hline \hline
        \# CPUs & 2\\ \hline
    CPU & Intel\textsuperscript{\textregistered} Xeon\textsuperscript{\textregistered} E5-2670v2\\ \hline 
    Architecture & Ivy Bridge \\ \hline 
    \# Cores per CPU & 10 \\ \hline 
    \# Threads per CPU & 20\\ \hline 
    Clock Freq. & 2.5 GHz\\ \hline \hline 
    L1 Cache & 320KB \newline 32KB per core\\ \hline 
    L2 Cache & 2560KB  \newline  256KB per core \newline\\ \hline 
    L3 Cache & 25600KB \newline shared \\ \hline \hline 
    Inst. Set Ext. & SSE4.2 \& AVX \\ \hline 
        \#Memory Channels & 4\\ \hline \hline

    Vendors Announced Peak Memory BW & 59.7 GB/s\\ \hline
    Measured\footnote{Stream Benchamrk} Peak Memory BW & 58.5GB/s\\ \hline
  \end{tabular}
     \caption{Architectural characteristics of the evaluation platform.}
     \label{table:characterization}
\end{table}

The software used for both relational and linear algebra, and the corresponding versions are stated bellow:

\begin{itemize}
\item Linear Algebra: 
    \begin{itemize}
    \item Compiler: ICC version 16.0.0 (GCC version 4.4.6 compatibility)
    \begin{itemize}
        \item no vectorization: -O3 -std=c99 -no-vec -farray-notation 
        \item vectorization: -O3 -std=c99 -farray-notation -xAVX -vec-report7
    \end{itemize}
    \item Intel\textsuperscript{\textregistered} MKL Version	11.3
    \begin{itemize}
        \item Link line: -lmkl\_intel\_lp64 -lmkl\_core -lmkl\_sequential -lpthread -lm
    \end{itemize}
        \end{itemize}

\vspace{0.35cm}
    \item Relational Algebra (PostgreSQL version 9.6+);
    \begin{itemize}
        \item Built with the following dependencies:
        \begin{itemize}
            \item GCC version 4.9.0
            \item Python 2.6.6
        \end{itemize}
           \end{itemize}
\end{itemize}

PostgreSQL was compiled specifically for the test platform, to fully take advantage of the available computing resources. 

\subsection{Tuning the relational algebra engine}
Database, application, and storage servers ship with a large number of configuration parameters like buffer cache sizes, number of I/O daemons, and parameters input to the database query optimiser. Finding good settings for these parameters is a challenging task because of the complex ways in which parameter settings can affect performance. The parameters shared\_buffers, effective\_cache\_size, and work\_mem, were adjusted accordingly, with  shared\_buffers begin set to 2GB, effective\_cache\_size being set to 64GB, and work\_mem being set to 25MB. In order to further speed up RA queries, indexes were created for all the database tables.


\subsection{Tuning sparse CSC and CSR methods to assist data level parallelism}

Efficient SSE vectorisation was achieved  in both versions of the linear algebra approach. The usage of the performant Intel Math Kernel Library (MKL) fully assisted vectorisation of the dot product between sparse matrices and sparse matrix vector, on the CSR version.\par 
The compiler was also instructed via auto vectorisation hints, and user mandated vectorisation. The non existence of vector dependencies, when verified, was also explicitly included.\par 
Alignment of data and data structures can affect performance. 
In the memory allocation alignment all data was aligned  according to cache line size. Regarding the access alignment,
for AVX, alignment to 32-byte boundaries (8 SP chunks) allowed a single reference to a cache line to move 8-SP numbers into the registers. 
The compiler was instructed to to assume that all CSC and CSR arrays are aligned on an 32-byte boundary.

\subsection{Tuning RA and LA approaches}


We conducted experiments on the simplified TPC-H  query-1, shown in listing \ref{used_query_1}, similar to the one presented in listing \ref{query_1}. 
The experiments focused a variety of datasets ranging from 1GB to 32GB. An overview of their characteristics appears in table \ref{table:dataset_info}. 

The CSC format requires a  larger  overall space for data, but it allowed us to simplify several algorithms and improve the overall perfomance.

\lstinputlisting[caption=SQL code for the simplified TPC-H query 1 used on the experimentation, label=used_query_1]{sql/used_query_1.sql} %input de um ficheiro





\end{multicols}

\noindent

\begin{table}[H]
\centering
  \footnotesize
     \caption{Overview of the produced sparse matrices used in evaluation study.}
  \begin{tabular}{ | L{0.75cm} | L{3.25cm} | L{3.25cm} | L{3.25cm} | L{3.25cm} | L{1.1cm} | L{1.1cm} | }
  
    \hline
    
  \multirow{2}{*}{Dataset} 	&	Measure Matrix Quantity			&	Projection Matrix return flag			&	Projection Matrix line status			&	Projection Matrix ship date			&	\multicolumn{2}{| L{2.4cm} |}{Overall SP Space Required}			  \\ \cline{2-7}

	&	Dimensions, Nonzeros, CSR space, CSC space			&	Dimensions, Nonzeros, CSR space, CSC space			&	Dimensions, Nonzeros, CSR space, CSC space			&	Dimensions, Nonzeros, CSR space, CSC space			&	CSR format	&	CSC format	  \\ \hline
1	& (	6001215	$\times$	6001215	) & (	3	$\times$	6001215	) & (	5	$\times$	6001215	) & (	2531	$\times$	6001215	) &		&		  \\ \cline{2-5}
	&	nnz:		6001215	&	nnz:		6001215	&	nnz:		6001215	&	nnz:		6001215	&		&		  \\ \cline{2-5}
	& CSR:	69	 MB CSC: 	69	MB & CSR:	46	 MB CSC: 	69	MB & CSR:	46	 MB CSC: 	69	MB & CSR:	46	 MB CSC: 	69	MB &	206	MB &	275	MB  \\ \hline
2	& (	11997996	$\times$	11997996	) & (	3	$\times$	11997996	) & (	5	$\times$	11997996	) & (	2531	$\times$	11997996	) &		&		  \\ \cline{2-5}
	&	nnz:		11997996	&	nnz:		11997996	&	nnz:		11997996	&	nnz:		11997996	&		&		  \\ \cline{2-5}
	& CSR:	137	 MB CSC: 	137	MB & CSR:	92	 MB CSC: 	137	MB & CSR:	92	 MB CSC: 	137	MB & CSR:	92	 MB CSC: 	137	MB &	412	MB &	549	MB  \\ \hline
4	& (	23996604	$\times$	23996604	) & (	3	$\times$	23996604	) & (	5	$\times$	23996604	) & (	2531	$\times$	23996604	) &		&		  \\ \cline{2-5}
	&	nnz:		23996604	&	nnz:		23996604	&	nnz:		23996604	&	nnz:		23996604	&		&		  \\ \cline{2-5}
	& CSR:	275	 MB CSC: 	275	MB & CSR:	183	 MB CSC: 	275	MB & CSR:	183	 MB CSC: 	275	MB & CSR:	183	 MB CSC: 	275	MB &	824	MB &	1098	MB  \\ \hline
8	& (	47989007	$\times$	47989007	) & (	3	$\times$	47989007	) & (	5	$\times$	47989007	) & (	2531	$\times$	47989007	) &		&		  \\ \cline{2-5}
	&	nnz:		47989007	&	nnz:		47989007	&	nnz:		47989007	&	nnz:		47989007	&		&		  \\ \cline{2-5}
	& CSR:	549	 MB CSC: 	549	MB & CSR:	366	 MB CSC: 	549	MB & CSR:	366	 MB CSC: 	549	MB & CSR:	366	 MB CSC: 	549	MB &	1648	MB &	2197	MB  \\ \hline
16	& (	95988640	$\times$	95988640	) & (	3	$\times$	95988640	) & (	5	$\times$	95988640	) & (	2531	$\times$	95988640	) &		&		  \\ \cline{2-5}
	&	nnz:		95988640	&	nnz:		95988640	&	nnz:		95988640	&	nnz:		95988640	&		&		  \\ \cline{2-5}
	& CSR:	1099	 MB CSC: 	1099	MB & CSR:	732	 MB CSC: 	1099	MB & CSR:	732	 MB CSC: 	1099	MB & CSR:	732	 MB CSC: 	1099	MB &	3296	MB &	4394	MB  \\ \hline
32	& (	192000551	$\times$	192000551	) & (	3	$\times$	192000551	) & (	5	$\times$	192000551	) & (	2531	$\times$	192000551	) &		&		  \\ \cline{2-5}
	&	nnz:		192000551	&	nnz:		192000551	&	nnz:		192000551	&	nnz:		192000551	&		&		  \\ \cline{2-5}
	& CSR:	2197	 MB CSC: 	2197	MB & CSR:	1465	 MB CSC: 	2197	MB & CSR:	1465	 MB CSC: 	2197	MB & CSR:	1465	 MB CSC: 	2197	MB &	6592	MB &	8789	MB  \\ \hline
  \end{tabular}
     \label{table:dataset_info}
\end{table}
\vspace{2cm}

\begin{multicols}{2}


\subsection{Experimental results analysis}

Figure \ref{fig:time_la_vs_ra} plots the measured execution for a range of datasets, and both linear and relational algebra approaches.

Denote that the presented values were selected through the K-Best technique, with K=3, from 50 samples. 

For the linear algebra approach we present the best solution from CSR and CSC format. The presented linear algebra solution in figure \ref{fig:time_la_vs_ra} is the one using CSR and format and  Intel MKL version 11.3. \par
The experiment presented in the current section focus on the best possible sequential solutions for both relational and linear algebra versions. The parallel PostgreSQL line was only introduced to elucidate the parallel  goal to achieve in the LA versions discussed in later sections of this report, and discuss future potential improvements of our linear algebra system. 

As shown, the coding effort to tune sparse CSR methods to assist data level parallelism revealed itself fruitful. However, a further analysis should be produced in order to fully potentiate vectorisation opportunities and optimisations.\par 

\begin{figure}[H]
\centering
\caption{Execution time for the  simplified query-1 from TPC-H benchmark, for different scale factors (dataset sizes from 1GB to 32GB).}
\includegraphics[width=1\columnwidth]{eps/TIME_LA_vs_RA_1st.eps}
\label{fig:time_la_vs_ra}
\end{figure}





\section{Parallelisation}
\label{parallel}
\indent
\par 

\subsection{Sequential Profiling}
Prior to parallelisation every step of the linear algebra expression was profiled in order to identify potencial hotspots. Since in the later sections, the best parallel linear algebra version turned out to be the CSC version, special attention is added to it when compared to the CSR version. \par 
Table \ref{table:profile_seq} demonstrates the profiler results, namely the percentage of overall time for each of the operation present in expression \ref{eq:tpch_1}, for the CSC version, for the largest TPC-H dataset in test - 32GB.

\begin{table}[H]
\centering
\footnotesize
  \begin{tabular}{ | L{1cm} | L{1.25cm} |  L{1.2cm} |  L{1.3cm} |  L{1.6cm} | L{1cm} |  }
    \hline
    LA Version	&	Projection	&	Selection	&	Projection . Selection	&	(Projection .Selection). Quantity	&	Bang	\\ \hline
Seq. CSC	&	23.93\%	&	43.38\%	&	10.89\%	&	18.12\%	&	3.68\%	\\ \hline
  \end{tabular}
     \caption{Profiling results for the sequential CSC linear algebra version, for TPC-H 32GB dataset, for the evaluation platform.}
     \label{table:profile_seq}
\end{table}

As you can observe  on table \ref{table:profile_seq}, the most time consuming operations are the Selection and the Projection (Khratri-Rao operation). Further efforts will be made in order to reduce the selection overhead in the later subsection \ref{optimization_selection}. \par 

\subsection{Parallel Experimental Results and Analysis}

To parallelise both linear algebra versions, OpenMP version 4.0 was used.
Based on the CSC Sparse Format linear algebra version, since each column has at most one element, the used algorithms give place to good parallelisation opportunities. Regarding the CSR format the used algorithms present more challenging parallel  optimisations. \par 

Figure \ref{fig:time_la_vs_ra_parallel} presents the measured time analysis regarding the variety of datasets, the test platform, and both linear and relational algebra approaches. Denote that the presented values were selected through the K-Best technique, with K=3 from a 50 samples. Regarding the linear algebra approach we present the the best solution from CSR and CSC format, namely the CSC format version.\par
As visible, the parallel linear algebra solution surpasses the parallel relational algebra, however larger TPC-H datasets reduce the gain from the linear algebra approach. Further algorithm optimisation will be analysed in the following subsection \ref{optimization_selection}.\par 

\begin{figure}[H]
\centering
\caption{TPC-H benchmark simplified query-1 time for solution analysis for different scale factors, between parallel linear and relational algebra approaches.}
\includegraphics[width=1\columnwidth]{eps/TIME_LA_vs_RA_parallel.eps}
\label{fig:time_la_vs_ra_parallel}
\end{figure}

\subsection{Further Selection Algorithm Optimisation}
\label{optimization_selection}

Further profiling was made in order to fully examine the potential bottlenecks. Table \ref{table:profile_par} compares both sequential and parallel linear algebra versions. As you can observer the Projection operation has diminished the percentage of overall time to complete the operation. However the Selection operation has increased its percentage overall time to almost half the time. \par 


\begin{table}[H]
\centering
\footnotesize
  \begin{tabular}{ | L{1cm} | L{1.25cm} |  L{1.2cm} |  L{1.3cm} |  L{1.6cm} | L{1cm} |  }
    \hline
    LA Version	&	Projection	&	Selection	&	Projection . Selection	&	(Projection .Selection). Quantity	&	Bang	\\ \hline
Seq. CSC	&	23.93\%	&	43.38\%	&	10.89\%	&	18.12\%	&	3.68\%	\\ \hline
Par. CSC	&	13.62\%	&	46.83\%	&	11.49\%	&	16.74\%	&	3.51\%	\\ \hline

  \end{tabular}
     \caption{Profiling results for the parallel CSC linear algebra version, for TPC-H 32GB dataset, for the evaluation platform.}
     \label{table:profile_par}
\end{table}

After a careful examination of a portion the C source code for the CSC version selection, presented on listing \ref{selection_code_old}, further algorithm optimisations can be made. 

\lstinputlisting[caption=portion of the C source code for the CSC linear algebra version selection algorithm, label=selection_code_old]{src/selection_old.c} 

Taking as example the TPC-H 32GB dataset, and the shipdate matrix, with dimension $(\ m\ \times\ n\ ), namely $(\ 2531\ $\times\ 192000551\ )$, to compute the selection result, in the worst case scenario, it is necessary to realize $(n * 2) = 384001102 $ string comparisons. However, by analysing table \ref{table:dataset_info}, you can denote that from the 384001102 string comparisons only 2531 strings will be tested.\par 
 Listing \ref{selection_code_new} avoids repetitive string comparisons  adding and additional auxiliar array (of size n), resulting in a worst case scenario of  $(m * 2) = 5062 $ string comparisons and $n = 192000551 $ integer comparisons.\par
 This solution resolves two distinct problems: the first being the excessive overhead of requiring a large amount and repetitive string comparisons, and the second being the improvement of this solution with bigger datasets. The additional overhead of producing an auxiliar array gets diminished with the increase of the dataset  and consequently the required comparisons.  The bigger the dataset the better. 

\lstinputlisting[caption=portion of the C source code for the CSC linear algebra version selection algorithm, label=selection_code_new]{src/selection_new.c} 

Table \ref{table:profile_par_new} compares both sequential, parallel, and optimised parallel linear algebra versions. As you can observer the Selection operation has largely diminished its percentage overall time making it possible for the overall time to be averagely distributed among operations. \par 

\begin{table}[H]
\centering
\footnotesize
  \begin{tabular}{ | L{1cm} | L{1.25cm} |  L{1.2cm} |  L{1.3cm} |  L{1.6cm} | L{1cm} |  }
    \hline
    LA Version	&	Projection	&	Selection	&	Projection . Selection	&	(Projection .Selection). Quantity	&	Bang	\\ \hline
Seq. CSC	&	23.93\%	&	43.38\%	&	10.89\%	&	18.12\%	&	3.68\%	\\ \hline
Par. CSC	&	13.62\%	&	46.83\%	&	11.49\%	&	16.74\%	&	3.51\%	\\ \hline
Optimized Par. CSC	&	10.47\%	&	20.10\%	&	25.86\%	&	34.60\%	&	8.98\%	\\ \hline
  \end{tabular}
     \caption{Profiling results for the selection algorithm optimised parallel CSC linear algebra version, for TPC-H 32GB dataset, for the evaluation platform.}
     \label{table:profile_par_new}
\end{table}

\subsection{Final Parallel Experimental Results}

Figure \ref{fig:time_la_vs_ra_parallel_v2} presents the measured time analysis regarding the variety of datasets, the test platform, and both linear and relational algebra approaches. Denote that the presented values were selected through the K-Best technique, with K=3 from a 50 samples. 
Regarding the linear algebra approach we present the the best solution from CSC format, with the selection algorithm optimisation.\par
As visible, the parallel linear algebra solution surpasses in both versions the parallel relational algebra, and the problem regarding the reduction of gain with the increase of  TPC-H datasets was eliminated.\par 
Figure \ref{fig:speedup_la_vs_ra_parallel} states the obtained speedup between the best linear algebra version and the best relation algebra version.

\begin{figure}[H]
\centering
\caption{TPC-H benchmark simplified query-1 time for solution analysis for different scale factors, between parallel linear and relational algebra approaches, in which linear algebra approach states the selection algorithm optimisation.}
\includegraphics[width=0.95\columnwidth]{eps/TIME_LA_vs_RA_parallel_v2.eps}
\label{fig:time_la_vs_ra_parallel_v2}
\end{figure}



\begin{figure}[H]
\centering
\caption{TPC-H benchmark simplified query-1 speedup analysis for different scale factors, between sequential linear algebra vs parallel linear algebra and parallel linear algebra vs parallel relational algebra.}
\includegraphics[width=0.95\columnwidth]{eps/speedup.eps}
\label{fig:speedup_la_vs_ra_parallel}
\end{figure}




\section{Conclusion}
\indent
\label{conclusion}

We have designed, implemented and evaluated an efficient and parallel LA framework to provide fast responses to OLAP queries in very large databases. We started by developing a typed linear algebra solution to respond to OLAP TPC-H query 1, with faster response times than a Open Source relational algebra competitor, PostgresSQL. 

A challenge in this project was to understand the linear algebra theory prior defined\cite{macedo2015linear} \cite{Po15}, to fully represent relational algebra in terms of linear algebra operators. This process was longstanding and it was only managed to do it with help of our advisors. Efforts were made to correctly introduce the key ideias prior presented\cite{macedo2015linear}, and resolving potential difficulties that the previous implementations faced. \par


%N�O SEI O QUE O PROFESSOR QUER COM A LINHA AO LADO DESTE PARAGRAFO
In spite the already 18x speedup regarding sequential linear algebra version and the 2x speedup regarding the parallel relational algebra engine, a lot of work can be done to further improve the algorithm, and fully analyse TPC-H queries results, as stated on section \ref{future_work}, leaving several potential exploratory paths.

 





\section{Future Work}
\label{future_work}
\indent


The biggest problem of the linear algebra implementation relies in the fact that the intermediate matrices, due to the possibility of a column might not have a nonzero element, are not optimal for further parallelisations. This is the unique speedup container regarding the relational algebra approach. Surpassed that limitation the speedups would ascend to more significant values when compared to the relational algebra version. \par 
It is also clear that the keys for selection significantly interfere with both linear and relational algebra algorithms. In the LA specific case, a selection operation that returns a low number of nonzero elements might compromise the attainable speedup through parallelisation, since there might not be enough data de keep the computational units busy. \par 

There is still room to greatly improve the parallelisation of
this algorithm. For example,  a Gather/Scatter analysis for distributed memory parallelism might result in larger attainable speedup, however, efforts need to me made to detect possible latency or bandwidth issues. \par 
Another approach resides on the duplication of data in a shared memory environment . For the CSC format, regarding the three arrays, every thread would have the total CSC column pointer array and a portion of CSC values and CSC row indexes arrays. The computation of both divided arrays would be thread independent and there would only be necessary to reduce the CSC column pointer. \par 

In order to improve data locality the Block Compressed Sparse Row Format could be used. However the BSR format would largely increase algorithm complexity for the defined methods. That improvement should only be used in the CSR linear algebra defined version.\par 

Future work also includes extending the scope of both offline and run-time optimisations. These include:
\begin{itemize} 
\item investigating reordering methods to reduce total query compute time.
\item exploiting index compression, further cache-blocking and TLB blocking to reduce memory traffic and to further improve locality.
\item implementing other matrix partitioning schemes.
\item improving load balance when there are different data structures for each generated matrix.
\item reducing data structure conversion costs at run-time for the CSR linear algebra version.
\item determining the most efficient  number of cores for parallelism.
\item fully translate the remaining TPC-H queries and benchmark both linear and relational algebra approaches.
\end{itemize}

The usage of CUDA and MIC systems should also be explored as an heterogenous system solution. Since the algorithms require huge portions of simple computation, and it is already parallelised via OpenMP, porting the application to be Xeon Phi compatible should not present great challenges. That solution was not explored despite the simplicity of the porting mainly because of a new counterpart -- the smart scheduling of the multi-architecture processing units.\par 


 


\section{Acknowledgements}
\indent


\cite{macedo10matrices}
Aggregation Performance.







Database, application, and storage servers ship with a large number of configu- ration parameters like buffer cache sizes, number of I/O daemons, and parameters input to the database query opti- mizer?s cost model. Finding good settings for these param- eters is a challenging task because of the complex ways in which parameter settings can affect performance


 
%---------------------------------------------------------------------------------------
%	REFERENCE LIST
%--------------------------------------------------------------------------------------

\bibliography{biblio}

\bibliographystyle{plain}

%---------------------------------------------------------------------------------------
\end{multicols}
\end{document}



